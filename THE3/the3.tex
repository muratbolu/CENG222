\documentclass[12pt]{article}
\usepackage[utf8]{inputenc}
\usepackage{float}
\usepackage{amsmath}


\usepackage[hmargin=3cm,vmargin=6.0cm]{geometry}
\topmargin=-2cm
\addtolength{\textheight}{6.5cm}
\addtolength{\textwidth}{2.0cm}
\setlength{\oddsidemargin}{0.0cm}
\setlength{\evensidemargin}{0.0cm}
\usepackage{indentfirst}
\usepackage{amsfonts}

\begin{document}

\section*{Student Information}

Name : Murat Bolu \\

ID : 2521300 \\


\section*{Answer 1}
\subsection*{a)}

Since the sample is small and the standard deviation of the population is
unknown, Student's \textit{t} distribution can be used. The critical value of
\textit{t} distribution is $\textit{t}_{\alpha/2} = \textit{t}_{0.01} = 2.602$
with $n-1=15$ degrees of freedom.

\begin{align*}
    \bar{X}&= \frac{8.4+7.8+6.4+6.7+6.6+6.6+7.2+4.1+
                    5.4+6.9+7.0+6.9+7.4+6.5+6.5+8.5}{16} \\
    &= 6.806
\end{align*}

\begin{align*}
    s &= \sqrt{\frac{\sum_{i=1}^{n}(X_i - \bar{X})^2 }{n-1}} \\
    &= \sqrt{\frac{\sum_{i=1}^{16}(X_i - 6.806)^2}{15}} \\
    &= 1.055
\end{align*}

The 98\% confidence interval $\bar{X} \pm \textit{t}_{\alpha/2}
\frac{s}{\sqrt{n}}$ becomes $6.806 \pm 2.602 \cdot \frac{1.055}{4} =
[6.120,7.493]$.

\subsection*{b)}

Let $\mu_1$ be the initial gasoline consumption per 100 kilometers and $\mu_2$
be the improved gasoline consumption per 100 kilometers. Therefore, the null
hypothesis $H_0$ is $\mu_1 = \mu_2$ and the alternative hypothesis $H_A$ is
$\mu_1 > \mu_2$. Since we don't know the population standard deviation, we can
use T-statistic.

\begin{align*}
    \textit{t} &= \frac{\bar{X} - \mu}{s / \sqrt{n}} \\
    &= \frac{6.806 - 7.5}{1.055 / 4} \\
    &= -2.629
\end{align*}

The rejection region is $R = (-\infty, -\textit{t}_{\alpha}] = (-\infty,
-\textit{t}_{0.05}] = (-\infty, -1.753]$ with 15 degrees of freedom, since we
are using a left-tail alternative. Since $t \in R$, we can reject the null
hypothesis. There is sufficient evidence that the improvement was effective.

\subsection*{c)}

Since $\bar{X} = 6.806 > 6.5$, we can immediately accept the null hypothesis because the T-statistic is positive and the rejection region includes zero.

\section*{Answer 2}
\subsection*{a)}
\subsection*{b)}
\subsection*{c)}
\subsection*{d)}


\section*{Answer 3}

\section*{Answer 4}

\end{document}
